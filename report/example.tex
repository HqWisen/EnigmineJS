\documentclass[letterpaper]{article}
\usepackage{natbib,alifexi}

\title{Critical Exponent of Species-Size Distribution in Evolution}
\author{Chris Adami$^{1}$, Ryoichi Seki$^{1,2}$ \and Robel Yirdaw$^2$ \\
\mbox{}\\	
$^1$California Institute of Technology, Pasadena, CA 91125 \\
$^2$California State University, Northridge, CA 91330 \\
adami@caltech.edu}


\begin{document}
\maketitle

\begin{abstract}
  We analyze the geometry of the species-- and genotype-size
  distribution in evolving and adapting populations of single-stranded
  self-replicating genomes: here programs in the Avida world.  We find
  that a scale-free distribution (power law) emerges in complex
  landscapes that achieve a separation of two fundamental time scales:
  the relaxation time (time for population to return to equilibrium
  after a perturbation) and the time between mutations that produce
  fitter genotypes. The latter can be dialed by changing the mutation
  rate.  In the scaling regime, we determine the
  critical exponent of the distribution of sizes and strengths of
  avalanches in a system without coevolution, described by first-order
  phase transitions in single finite niches.
\end{abstract}

\section{Introduction}

Power law distributions in Nature usually signal the absence of a
scale in the region where the scaling is observed, and sometimes point
to critical dynamics. In Self-Organized-Criticality (SOC)
\citep{BTW87,BTW88}
\section{Acknowledgements}

This work was supported by NSF grant No.\ PHY-9723972.

\footnotesize
\bibliographystyle{apalike}
\bibliography{enigmine}


\end{document}
